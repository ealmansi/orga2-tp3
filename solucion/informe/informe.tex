% ** Cuidado: El indice se actualiza al compilar 2 veces. **

% -- Configuración --
\documentclass[10pt, a4paper]{article}
\usepackage[paper=a4paper, left=1.5cm, right=1.5cm, bottom=1.5cm, top=1.5cm]{geometry}
\usepackage[utf8]{inputenc}
\usepackage[spanish]{babel}
\usepackage{float}
\usepackage{mdframed}
\usepackage[section]{placeins}
% \usepackage[colorlinks=true, linkcolor=blue]{hyperref}
\usepackage{graphicx}
\usepackage{titling}
\usepackage{verbatim}
\usepackage{tikz}
\usepackage{dsfont}
\usepackage{amsmath}
\usetikzlibrary{chains,fit,shapes}
\newcommand{\ig}[1]{\includegraphics[width=\textwidth]{#1}}
\newcommand{\RNum}[1]{\uppercase\expandafter{\romannumeral #1\relax}}
\usepackage{parskip}
\usepackage{handy}
\usepackage{mdframed}

 
% -- Documento --
\begin{document}

\setlength{\parindent}{25pt}

	% -- Carátula --
	
	\thispagestyle{empty}
	\title{%
	\huge{Organización del Computador \RNum{2}}\\
	\vspace{4mm}
	\large{Trabajo Práctico 2}
	}
	\date{\vspace{5mm}$2^{\mathrm{do}}$ cuatrimestre de 2013}

	\author{
		\\
		{\rm Emilio Almansi }\\
		\small{ealmansi@gmail.com}
		\and
		\\
		{\rm Álvaro Machicado }\\
		\small{rednaxela.007@hotmail.com}
		\and
		\\
		{\rm Miguel Duarte}\\
		\small{miguelfeliped@gmail.com}
	} % end author
	\maketitle

	
	\vspace{10mm}
	% -- Resumen --
	\begin{abstract}
		\centering{%
			Un breve abstract sobre como arruinar tu vida en 24 hs, cortesía de Bochs.
		}
	\end{abstract}
	\vspace{10mm}

	% -- Indice --

	\tableofcontents

	\clearpage
	% -- Contenido --
	% Se edita cada seccion por separado, no hay que cambiar nada acá.
	\section{Introducción}
		
Se desarrolló un kernel elemental para la arquitectura Intel x86 con soporte básico para las unidades de segmentación y paginación, manejo de interrupciones y cambio automático de tareas por hardware.

El alcance del trabajo incluye múltiples etapas del proceso de arranque de todo sistema operativo, permitiendo apreciar las herramientas que provee el procesador para la administración del sistema. Durante la etapa de incialización, es necesario llevar a cabo distintos pasos como el traspaso de modo real a modo protegido, la habilitación de la unidad de paginación, o el armado de múltiples estructuras requeridas por el procesador para el manejo de interrupciones o de tareas.

Adicionalmente, se implementó un scheduler rudimentario permitiendo realizar la ejecución por tiempo compartido de múltiples tareas definidas al tiempo de compilación, realizando un ciclo de ejecución no lineal guiado por tics del reloj.

	\clearpage

	\section{Global Descriptor Table (GDT)}
			LA GDT (\textit{Global Descriptor Table})  es una estructura de datos
que la arquitectura intel x86 utiliza para almacenar distintos descriptores
de sistema. En este trabajo sólo utilizamos 3 tipos de descriptores: El
primero y mas trivial es un descriptor nulo. El procesador produce
una excepción automáticamente cunado se intenta acceder a la posición cero
de la GDT. De esta manera el primer espacio de la tabla se rellena con
un descriptor nulo para evitar confusiones. Es decir que este descriptor
en realidad cumple una función accesoria, no modifica en nada el funcionamiento
del sistema. Sin embargo es muy cómodo que este ahí. El segundo tipo
de descriptor le corresponde a los descriptores de segmento. Los mismos
delimentan el tamaño y la ubicación de los segmentos, así como sus propiedades
(quién los puede acceder, que clase código hay adentro, etc, etc. Por último
se utilizaron descriptores de TSS para realizar el salto automático de tareas
(sobre esto vamos a hablar mas adelante con mayor profundidad, ahora sólo
se explicará la parte que concierne específicamente a la GDT).


\subsection{Segmentación flat}

	Como indica el enunciado todo el trabajo funciona con segmentación flat.
Esto significa dejar de lado la protección por segmentación armando 4 segmentos
que ocupen toda la memoria superpuestos entre si. Estos 4 segmentos son 2
con nivel de privilegio cero (máximo nivel de provilegios) y 2 con nivel de privilegio
3 (mínimo nivel de privilegios), y para cada nivel un segmento de código y uno
de datos.

	Por supuesto que esta práctica acarrea problemas, algunos de ellos graves
como por ejemplo que se pierde toda la protección por hardware brindada
por la segmentación. La contracara de esto es que se obtiene un entorno
mucho mas amistoso para programar, y la mayoría de la seguridad que se pierde
por la segmentación se puede recuperar utilizando de manera adecuado la paginación.

\subsection{Descriptores de segmento}

	Los descriptores de segmento se hardcodean en tiempo de compilación. Es decir
que en el mismo código la GDT ya contiene sus descriptores de segmento con
los parámetros adecuados. Es importante que esto se puede realizar exclusivamente
porque todos lo necesario para completar esos descriptores se conoce de antemano.
Esto en parte es una consecuencia de la segmentación flat.

	Una vez que el programa empieza a correr estos descriptores de segmento
no se vuelven a modificar. Se trabaja siempre los segmentos flat, por lo que no
hace falta modificar ni agregar nada. Quedan estáticos para siempre.

	Además de los segmentos flat se crea un segmento que contiene la memoria de video.
Este segmento se utilizó con fines didácticos en un principio, pero luego
todas las funciones de video que se utilizan a lo largo del trabajo acceden
a la memoria de video por medio del segmento flat de datos de nivel 0.

\subsubsection{Atributos descriptores de segmento}


\begin{figure}[h]
\begin{center}
  \includegraphics[scale=0.3]{secciones/dibujitos/descriptorDeSegmento.png}
\end{center}
\caption{Descriptor de segmento}
\label{fig:descriptorDeSegmento}
\end{figure}

	
	En realidad esta es posiblemente la parte del trabajo donde menos libertad
existe. Al usar sementación flat sólo existe una posibilidad correcta en los
atributos para cada segmento. La base se tiene que estar en cero, 
el límite en el tamaño máximo de cada segmento (1,75gb en este caso),
G tiene que estar en 1, pues los sectores tienen que ser de 4kb,
y P tiene que estar en 1, pues el segmento tiene que figurar como
presente siempre.

	El resto de los parámetros varían en cada segmento, pero siempre
sin absolutamente nada de margen. 2 segmentos tienen que ser de sistema
y tener DPL 0, otros 2 tienen que ser de usuario y tener DPL 3. Por último
un segmento de sistema tiene que ser de datos y otro de código y los
de usuario también tienen que ser uno de datos y otro de código.

\subsection{Descriptores de TSS}

	Para los descriptores de TSS se eligió otra dinámica. Las TSS se encuentran
ubicadas en un arreglo de TSS que se crea dinámicamente en tiempo de ejecución, por lo
que en el momento de la compilación no se sabe el lugar donde va a estar y por lo tanto
no se puede hardcodear esa dirección.

	Lo que se hizo, entonces, fue agregar los descriptores de TSSde manera dinámica en tiempo
de ejecución. Para esto se crearon 4 funciones en C:

\begin{minted}{c}
	void tss_inicializar_entrada_gdt_tarea_inicial();
	void tss_inicializar_entrada_gdt_idle();
	void tss_inicializar_entrada_gdt_navio(unsigned int nro_tarea);
	void tss_inicializar_entrada_gdt_bandera(unsigned int nro_tarea);
\end{minted}

	Si bien estas funciones tienen mucho en común se las creo por separado
por una cuestión pragmática que sirvió para encontrar errores y \textit{bugs} mas
fácilmente.

	Estas funciones a su vez se engloban todas en otra función escrita en c que
inicializa todas las tss y se llama desde \textbf{kernel.asm}


\begin{minted}{c}
	void tss_inicializar();
\end{minted}





	
\begin{comment}
	
	En la GDT hay que poner los descriptores de segmento
y los descriptores de TSS para cada cada tarea y para cada bandera.

	La misma está representada como un arreglo ``$gdt\_entry$'' declarado
de manera global en C. Las $gdt\_entry$ son structs de 4 bytes que poseen un campo
cara cada atributo de una entrada de gdt.

	Los descriptores de segmento fueron cargados de manera estática
en tiempo de compilación. Lo mismo con el descriptor de la IDT. Esto
fue posible porque se conocen de antemano todos los valores
necesarios para completar los descriptores.

	A la hora de cargar los descriptores de TSS nos encontramos con la
siguiente dificultad: Los descriptores de TSS fueron declarados como una
variable global en código C. Por lo tanto en tiempo de compilación
no se sabe en que dirección van a ser cargados. Por este motivo se cargan
de manera dinámica mediante una función que se llama desde kernel.asm. La
función sencillamente crea una entrada más en el arreglo que representa 
de $gdt\_entry$ con los atributos adecuados.
\end{comment}

	\clearpage

	\section{Paginación}
			La paginación es un método de direccionamiento de memoria
en el cuál se redirigen bloques de direcciones ``virtuales''
a bloques de direcciones físicas. Estos bloques se llaman páginas, y
en la arquitectura intel en el modo usado en el trabajo tiene
un tamaño de 4kb (\texttt{0x1000}). La gran ventaja de la paginación
es que la unidad en la que se trabaja la memoria (la \textit{página})
tiene siempre el mismo tamaño, lo cuál hace que sea fácil hacer manejos
complejos de memoria.

	En la arquitectura intel x86 la paginación se maneja mediante
una estructura de sistema en memoria en 2 niveles. El primer nivel se llama
\textit{page directory} y el segundo \textit{page table}. Los page directory
contienen las direcciones donde se ubican los page table y los page
table contiene direcciones físicas de páginas. De esta manera una dirección
virtual (es decir la dirección a la que accede una tarea) se interpreta
como un índice en el page directory, un índice en la page table obtenida y
, finalmente, un offset para la página encontrada en el page table.

\begin{figure}[h]
\begin{center}
  \includegraphics[scale=0.7]{secciones/dibujitos/mmu.png}
\end{center}
\caption{Estructura de paginaci\'on}
\end{figure}

	La siguiente gran ventaja de todo esto es que uno puede tener muchas
estructuras de paginación en memoria y asignarle una distinta a cada tarea.
Esto se hace mediante un registro de control, el \textbf{cr3}. Cuando la
paginación está activa cada vez que se realiza un acceso a memoria
el procesador usa el cr3 para encontrar la dirección del page dir. Modificando
el valor de cr3 se puede hacer que diferentes tareas tengan diferentes \textit{mapas de memoria}
, es decir que cuando accedan a iguales direcciones virtuales lleguen a
diferentes direcciones físicas.

	Si bien mas adelante se va a hablar sobre la seguridad
de manera específica es importante mencionar que debido a que la
segmentación flat detruye todos los sistemas de seguridad que provee
la segmentación la paginación se debe tratar con mucho cuidado, pues
este mecanismo tiene que asegurar toda la seguridad y la integridad de la memoria.

	Además el mapeo de páginas tiene que permitir que las tareas
se ejecuten como si su código estuviera en la posición 1GB (\texttt{0x40000000})
y puedan leer (pero no escribir) desde las direcciones \texttt{0x40002000-0x40002FFF}
direcciones de memoria del kernel.

	Para lograr todo esto implementamos un esquema de paginación en el cuál
existen páginas compartidas y páginas privadas. Pero además existen tablas
de páginas compartidas y tablas de páginas privadas. Además se tuvo especial
cuidado con los permisos otorgados a cada tabla para evitar accesos a memoria
indeseados.

\subsection{Implementación del mapa de memoria}
	El mapa de memoria se inmplementó tal cuál lo dice el enunciado. Hay
identity mapping en los primeros 8mb y después cada tarea tiene
los mapeos que necesita para acceder a su código y al ancla.
	
	En tiempo de ejecución se generan todas las estructuras necesarias
para que el sistema y las tareas puedan funcionar. Esto incluye un page directory para cada
tarea, 1 para el kernel, 1 page table privado de cada tarea (con
permisos de usuario) y 2 page table compartidos(con permisos de administrador).

	Las page table privadas en un principio se inicializan en cero, luego
se actualizan de manera adecuada con funciones de mapeo de página
de las que se hablará mas adelante.

	La otra acción que se realiza en este momento es la de
copiar el código de las tareas de la tierra al mar. Elegimos hacerlo
en este momento porque así ya se pueden hacer los mapeos y dejar
todo todo listo para el momento en que empicen a correr las tareas.

	Todo explicado recién se hacer mediante una sola función escrita en c
que luego se llama desde kernel. asm

	Los page table compartidos tienen los mapeos de las direcciones del kernel.
Esas direcciones estar mapeadas en todas las tareas y con los mismos atributos, motivo
por le cuál decidimos hacerlos comunes. Esto significa todas las primeras entradas
de page dir apuntan al mismo page table, el cuál redirecciona con permisos de administrador
a los primeros 4mb de kernel, del mismo modo todas las segundas entradas
de page dir apuntan también a una page table común que direcciona
a la segunda parte del kernel.


\begin{figure}[h]
\begin{center}
  \includegraphics[scale=0.3]{secciones/dibujitos/diagramapaginas.jpg}
\end{center}
\caption{Esquema de las estructuras de paginación}
\label{fig:diagramapaginas}
\end{figure}


	Una vez que el sistema está corriendo los mapeos de página se modifican
en tiempo real mediante dos funciones que se encargan de crear
mapeos de página y de borrar mapeos de página.

\begin{minted}{c}
void mmu_mapear_pagina (unsigned int virtual, unsigned int cr3, unsigned int fisica, unsigned int attr) 
void mmu_unmapear_pagina (unsigned int virtual, unsigned int cr3) 
\end{minted}


	La función encargada de deshacer los mapeos termina con realizando un flush
de la tlb para asegurar la coherencia de todas las estructuras de paginación (
al cambiar un mapeo puede pasar que el mapeo almacenado en la tlb no coincida
más con la realidad).

	Todo esto fue implementado en C. Tanto las funciones
encargadas de crear las estructuras como las funciones encargadas
de los mapeos. Lo único que se hace desde \texttt{kernel.asm} es
llamar a esas funciones en el momento adecuado.

\subsection{Activando la paginación}

	La paginación se activa mediante el bit de paginación, en el
registro \textbf{cr0}. Es indispensable que, en el momento en que se
activa ese bit, cr3 ya esté seteado de manera adecuada y al menos las
estructuras de paginación del kernel esten debidamenta cargadas en memoria.



\begin{comment}
	La parte de paginación se resolvió de manera bastante intuitiva.

	Se crearon funciones en C que se encargar de inicializar los directorios
de páginas del kernel y de las tareas. Un detalle importante de la implementación
es que tanto el kernel como las tareas comparten las primeras 2 tablas de páginas
de sus directores de páginas.

	Estas dos tablas son las que se hacen con identity mapping. El identity mapping
está en todos los mapas de memoria de la misma manera y con los mismos atributos. Además
nunca debe ser cambiado a lo largo de la ejecución de todo el programa en ninguno de los
mapas. Por eso decidimos crear sólo 2 tablas de páginas y hacer que todas las tareas lo compartan.

	En un princio a cada tarea se le asigna una tabla extra inicializada en cero.
Luego mediante las funciones para mapear páginas se completan estas tablas de manera adecuada
para que se efectivicen los mapeos.

	Es importante notar que el resultado final de esto es que cada tarea tiene mapeadas
2 tablas con identity mapping y con proviligios restringidos (sólo para supervisor) y
luego un par de páginas mas con permisos de usuario, que son donde efectivamente va a trabajar.

	Todo eso se englobó en las siguientes funciones de C:
	
\begin{minted}[tabsize=4]{c}
	void mmu_inicializar_dir_kernel();
	void mmu_inicializar_paginas_kernel();
	void mmu_inicializar_tareas();
\end{minted}

	$mmu\_inicializar\_tareas$ no solo inicializar los directorios de páginas sino
que además hace los mapeos de páginas correspondientes.

	Finalmente esas funciones se llaman dentro de kernel.asm. Una vez terminado eso
se habilita la paginación.

\end{comment}

	\clearpage

	\section{Manejo de tareas}
			El manejo de tareas tiene varias aristas. Por un lado está
la gestión ``física'' de las tareas. Es decir como se cambia entre el contexto
de una y el contexto de otra. Este punto se manejo mediante el mecanismo
automático que provee la arquitectura intel.

	Para esto hubo que crear tss para cada tarea. Además por cuestiones
de facilitar la gestión la tareas se decidió crear una tss extra para cada
bandera. De esta forma cada navio tiene 2 tss, una para correr su código
de acción y otra para correr su código de bandera. Además hay 2 tss
extra, una para la tarea idle y otra auxiliar (\emph{tarea inicial}), para poder realizar
el primer salto sin inconvenientes.

	Para poder realizar el primer cambio de contexto (\texttt{JMP FAR}) sobre esta
tss auxiliar, previamente se setea el TR. Este seteo está hardcodeado
porque el descriptor de la tss auxiliar siempre se guarda exactamente
en el mismo índice de la gdt.

	Todos los cambios de tarea se hacen mediante un \texttt{JMP FAR}. En ningún
momento se usa ningún otro mecanismo. El problema que surge con
esto es que las banderas no tienen un código cíclico. Es decir, las banderas
ejecutan una porción de código que termina, al terminar su contexto queda
estático en la posición en la que terminó, por lo tanto si se llama a esa bandera
otra vez en ese mismo contexto sin hacer nada la bandera va a ejecutar 
cosas indebidas. Para subsanar ese inconveniente siempre antes de saltar una
bandera se vuelve a inicializar su tss, de esta forma nos aseguramos de que el contexto
siempre sea el contexto inicial.

	

	\clearpage

	\section{Scheduler}
		
Dadas las dieciséis tareas del sistema -ocho navíos, ocho banderas-, el scheduler desarrollado ejecuta todas las tareas de forma itinerante, alternando entre ellas de la siguiente forma: se dedican tres ticks de reloj a ejecutar navíos, se procesan todas las funciones bandera pertenecientes a navíos activos, y luego se retoma la ejecución de los navíos. Los navíos se ejecutan secuencialmente por lo que todos reciben la misma cantidad de ticks mientras no sean desalojados.

Para implementar esta lógica, se realizaron colas circulares basadas en arreglos para los índices de la tabla GDT de los navíos y las banderas\footnote{De los descriptores de sus tablas TSS, para ser preciso.}. Como no se ingresan nuevos elementos dinámicamente, el mecanismo de elminación de una tarea se puede realizar sencillamente marcando su entrada con un valor inválido, como el 0 ya que nunca es un índice válido en la tabla.

Para mantener el orden de ejecución descripto, se guardaron adicionalmente contadores para la cantidad de navíos activos en la cola, la cantidad de navíos pendientes hasta alternar con la ejecución de banderas, e inversamente la cantidad de banderas pendientes.

De esta forma, se mantiene el invariante de que al menos uno de los dos contadores de tareas pendientes es nulo, y se consumen elementos de la cola opuesta hasta terminar la ronda. Es decir, inicializando la cantidad de navíos pendientes a 3 y la cantidad de banderas pendientes a 0, se dedicarán 3 ticks a ejecutar navíos, y luego se asigna la cantidad de tareas activas al contador de banderas pendientes, asegurando que se llamen todas las banderas disponibles.

Como ya se dijo, cuando una tarea necesita ser desalojada, su entrada correspondiente y la de su navío o bandera asociada se anulan, permitiendo evitar su ejecución simplmente salteando los ceros que se encuentren en la cola.

Finalmente, en el caso de que hayan sido desalojadas todas las tareas, el scheduler retorna un valor distinguido igual a 0, procediendo a ejecutarse la tarea idle initerrumpidamente de allí en adelante.
	\clearpage

	\section{Conclusiones}
			A lo largo de todo el trabajo se presentó la enorme cantidad de inconvenientes
que conlleva trabajar sobre la nada. Para realizar el trabajo fue imprecindible entender
lo que estaba sucediendo en el kernel a muy bajo nivel, pues constanemente surgen inconvenientes
cuya explicación no suele ser evidente.

	También una cosa que se presentó es que la dificultad de la implementación asciende al infinito
si uno lo permite. En todo momento se presentan ecrucijadas sobre como realizar algo específico teniendo
una libertad casi absoluta al tener permisos de kernel. En general en esos momentos intentamos optar
por la implementación mas clara con una lógica mas sólida perdiendo en algunos casos algo de performance.
Sin embargo lo que se ganó con eso fue un código mucho mas manejable.





\end{document}
