	Las tareas corren en el anillo de seguridad 3. El kernel en anillo 0.
A nivel paginación las tareas son usuario y el kernel es supervisor.

	Las tareas no pueden acceder a memoria del kernel bajo ninguna circinstancia.
Esto está asegurado en el mapa de memoria. El mismo si bien incluye las tablasde memoria
del kernel las incluye con un privilegios mayor, por lo tanto si una tarea intenta
acceder a esas páginas cae un page fault que termina en el desalojo de esa tarea.

	El kernel está programado para no tener que incurrir en ninguna excepción especial,
lo tanto el código que se ejecuta cuando ocurre una excepción es siempre el mismo. Desalojar
a esa tarea y saltar a la tarea iddle. Esto asegura que toda tarea que realice una acción
ilegal que produzca uan excepción (dividir entre cero, seg fault, page fault, etc) será
desalojada.

	Además hay un par de cuestiones extra. Por ejemplo, una bandera no puede llamar a una int0x50.
Si hace esto inmediatamente es desalojada.
