	A nivel de segmentación, el kernel se ejecuta en anillo de nivel 0 y las tareas 
en anillo de nivel 3. A nivel paginación, las tareas son usuario y el kernel es supervisor.

	Por construcción del mapa de memoria, las tareas no pueden acceder a memoria del 
kernel bajo ninguna circinstancia. Si bien el mapa incluye las tablas de memoria
del kernel, estas precisan un nivel de privilegios mayor (anillo 0); es decir, si una tarea intenta
acceder a esas páginas, se genera una falta en la unidad de paginación.

	Cuando ocurre una excepción o una tarea genera un error de cualquier tipo, el comportamiento 
implementado siempre es el mismo: desalojar a la tarea que generó el problema, y ejecutar la tarea 
idle hasta el próximo tick del reloj. Las razones adicionales por las cuales una tarea puede ser 
desalojada son: si esta hace un llamado a la inx0x66, si su función bandera hace un llamado a la int0x50, 
o si su bandera excede su cuota de tiempo de ejecución y aún no finalizó cuando se generá un nuevo tick del reloj.