
Se desarrolló un kernel elemental para la arquitectura Intel x86 con soporte básico para las unidades de segmentación y paginación, manejo de interrupciones y cambio automático de tareas por hardware.

El alcance del trabajo incluye múltiples etapas del proceso de arranque de todo sistema operativo, permitiendo apreciar las herramientas que provee el procesador para la administración del sistema. Durante la etapa de incialización, es necesario llevar a cabo distintos pasos como el traspaso de modo real a modo protegido, la habilitación de la unidad de paginación, o el armado de múltiples estructuras requeridas por el procesador para el manejo de interrupciones o de tareas.

Adicionalmente, se implementó un scheduler rudimentario permitiendo realizar la ejecución por tiempo compartido de múltiples tareas definidas al tiempo de compilación, realizando un ciclo de ejecución no lineal guiado por tics del reloj.
