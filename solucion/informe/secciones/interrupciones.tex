	Las interrupciones y las excepciones son el médotodo
a través del cuál se logra administrar las tareas y asegurar
la integridad del sistema. Todos los handlers de interrupción
fueron pensados y escritors con esos objetivos en mente.



\beggin{comment}
	Las interrupciones se manejaron de 2 maneras diferentes:

	Por un lado están las excepciones de sistema. De esas sólo se busca
que si una tarea las produce esta sea desalojada. De esta manera se pudieron
implementar mediante un macro sin mayores inconvenientes.

	Por otro lado están las otras interrupciones: La de reloj, la de teclado
y las syscalls. Estas si se tuvieorn que elaborarse de manera mas detenida.

	La interrupción de teclado basicamente es un gran $switch$ que verifica si
el contenido del código obtenido. Si este coincide con alguna de las teclas esperadas
(``e'',``m'', ``0-9'') entonces realiza la acción debida en pantalla.

	En las syscalls se tiene el cuidado de verificar quién las llama y como. Muchas
posibles llamadas a syscals son ilegales. Por ejemplo si la llamada a una int0x50 la realiza
una bandera esta debe ser desalojada. Lo mismo si esa interrupción es llamada con parámetros ilegales.
Toda esta lógica se implementó medante una función escrita en C y llamada desde assembler.

	La interrupción de reloj terminó siendo la mas compleja. Para implementarla lo que se hizo, entonces,
fue realizar una función en c que se llama desde la interrupción. Esta función verifica si quién
se estaba ejecutando al memento de la interrupción de reloj era una bandera. En ese caso la desaloja.
Luego le pide al scheduller el índice de la GDT que necesario pra realizar un JUMP far al contexto
de la otra tarea. A continuación se fija si la próxima tarea a ejecutarse es una bandera. Si es una bandera
entonces primero reinicia su tss y luego, si, realiza el JMP FAR.
\end{comment}
	
