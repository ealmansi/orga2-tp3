	Las interrupciones y las excepciones son el método a través del cuál 
se logra interactuar con agentes externos, asegurar la integridad del sistema 
ante la ocurrencia de errores, o administrar los recursos del sistema. Para cada 
posible interrupción que se desea atender es necesario definir un handler de 
interrupción, o rutina de atención, y declararla a través de un descriptor
en una estructura del sistema denominada Interrupt Descriptor Table (IDT).

	La IDT tiene espacio para hasta 1024 descriptores, donde cada uno está formado 
por 64bits y consta principalmente de un selector de un segmento de código, el offset 
dentro de ese segmento donde se encuentra la rutina de atención, y una serie de atributos 
que modifican el proceso de llamado al handler. Los atributos más significativos son 
el tipo de handler: trap gate, interrupt gate o task gate; el Descriptor Privilege Level,
o DPL, que determina el mínimo privilegio que debe tener el que llama a la interrupción; y 
el bit D que determina si son de 16bits o 32 bits.

	En el sistema implementado se realizaron esencialmente los siguientes tipos de 
hanlders de interrupción: aquellos donde predomina la lógica con respecto 
al control de los recursos de sistema, como la interrupción generada por el reloj 
(que marca la unidad de tiempo sobre la que funciona el scheduler), la rutina de atención 
que procesa los eventos generados por el teclado, los handlers de atención a las syscalls 
(implementando servicios para las tareas del sistema), y las rutinas que proveen la lógica 
destinada a asegurar la integridad del sistema como puede ante el caso de una excepción 
como por ejemplo un Page Fault o un Segmentation Fault.
	
	Para todas las interrupciones se escogió el tipo de descriptor como trap gate de 32-bits. 
La única variación se tomó en el DPL: los servicios del sistema o syscalls se configuraron con DPL igual a 3, y los demás handlers con DPL igual a 0, para que no se puedan acceder desde las tareas.

\subsection{Interrupciones de control de flujo y administración de sistema}

	La principal encargada de controlar el flujo de ejecución en el sistema es la int0x32: 
la interrupción generada por el reloj. Esta interrupción funciona de manera muy cercana 
al scheduler, consultándole cuál tarea es la próxima a ejecutar y realizando el cambio 
de tarea de ser necesario. El scheduler implementado tiene un valor de retorno distinguido: 
si retorna un valor igual a 0, significa que la tarea ejecutándose actualmente debe 
seguir haciéndolo, si por el contrario devuelve cualquier otro número se lo interpreta 
como un índice en la GDT al que se debe realizar un \texttt{jmp far}, suscitando un 
cambio de contexto automático. En general, el único caso donde hay que continuar ejecutando 
la tarea actual es al final del juego cuando solo permanece en ejecución la tarea idle.

	Por otra parte, antes de realizar el \texttt{jmp far} es importante
verificar si la tarea a la que se va a saltar es una bandera o un
navío. Antes de saltar a una bandera, se reinicializa su TSS para
que el EIP vuelva a apuntar al comienzo de la función. Si esto no se hiciera, 
la ejecución seguiría a partir de la interrupción de fin de bandera
y ese no es el compartamiento esperado. En los navíos, en cambio, sí se espera
que la ejecución continúe donde fue interrumpida.

	Es importante contemplar el hecho de que cuando se retoma la ejecución de una tarea, 
su EIP va a estar ubicado adentro de la interrupción donde se detuvo, por lo tanto tiene 
que siempre haber un JMP al final del llamado, incluso luego de un \texttt{jmp far}.
	
	Por último todos los handlers referidos a las excepciones de
sistema (page fault, etc) al final también cambian el flujo de ejecución
pues luego de la lógica referida a la seguridad tienen que retomar
el flujo de ejecución a algún lado, ya sea al lugar donde vino o
, en caso de que esa tarea haya sido desalojada, a algún otro lado
(la tarea idle).

\subsection{Interrupciones de seguridad}

	Las excepciones de sistema como \textit{General Protection}
son el mejor ejemplo. El algoritmo de estas interrupciones es
sencillo, lo que hace es verificar que quién produjo
esa excepción sea efectivamente una tarea. Si fue una tarea, la desaloja; de lo 
contrario, interrumpe al sistema y notifica que hubo una excepción. Luego
de desalojar la tarea, sencillamente salta hacia la tarea idle. Además, 
todas estas interrupciones tienen exactamente la misma lógica, por lo 
que en principio se podrían implementar todas con un macro de nasm.
El único código específico que se implementó para cada una es una funcion de C
que actualiza la pantalla de manera adecuada.


\begin{comment}
	Las interrupciones se manejaron de 2 maneras diferentes:

	Por un lado están las excepciones de sistema. De esas sólo se busca
que si una tarea las produce esta sea desalojada. De esta manera se pudieron
implementar mediante un macro sin mayores inconvenientes.

	Por otro lado están las otras interrupciones: La de reloj, la de teclado
y las syscalls. Estas tuvieron que elaborarse de manera mas detenida.

	La interrupción de teclado basicamente es un gran $switch$ que verifica si
el contenido del código obtenido. Si este coincide con alguna de las teclas esperadas
(``e'',``m'', ``0-9'') entonces realiza la acción debida en pantalla.

	En las syscalls se tiene el cuidado de verificar quién las llama y como. Muchas
posibles llamadas a syscals son ilegales. Por ejemplo si la llamada a una int0x50 la realiza
una bandera esta debe ser desalojada. Lo mismo si esa interrupción es llamada con parámetros ilegales.
Toda esta lógica se implementó medante una función escrita en C y llamada desde assembler.

	La interrupción de reloj terminó siendo la mas compleja. Para implementarla lo que se hizo, entonces,
fue realizar una función en c que se llama desde la interrupción. Esta función verifica si quién
se estaba ejecutando al memento de la interrupción de reloj era una bandera. En ese caso la desaloja.
Luego le pide al scheduler el índice de la GDT que necesario pra realizar un JUMP far al contexto
de la otra tarea. A continuación se fija si la próxima tarea a ejecutarse es una bandera. Si es una bandera
entonces primero reinicia su tss y luego, si, realiza el \texttt{jmp far}.
\end{comment}
	
