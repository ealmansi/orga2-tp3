	Las interrupciones y las excepciones son el método
a través del cuál se logra administrar los recursos del sistema y asegurar
la integridad del sistema. Todos los handlers de interrupción fueron
pensados y escritos con esos objetivos en mente.

	De esta manera en cada handler se pueden notar esas dos aristas,
en algunos handlers predomina la lógica con respecto al control de los
recursos de sistema, como en la interrupción de reloj que se maneja codo
a codo con el scheduller administrar el tiempo de ejecución entre
las distintas tareas. En otras, por el contrario, lo que mas se aprecia
es la lógica destinada a asegurar la integridad del sistema como puede
ser el caso de una excepción generada por un page fault o un seg fault.

\subsection{Interrupciones de control de flujo y administración de sistema}

	
	La principal encarga de esto es la int0x32, es decir la interrupción
que produce el reloj. Esta interrupción funcina de manera muy cercana al escheduller
y su función basicamente es saltar a la tarea a la que le toca ejecutar o
bien, no hacer nada en caso de que la tarea que está corriendo deba continuar.
Para tomar esta desición lo que hace es consultarle al scheduller que tarea
debe seguir corriendo. Si el escheduller devuelve 0 significa que la tarea
que corre actualmente debe seguir haciéndolo, si por el contrario
devuelve cualquier otro número se lo interpreta como un índice en
la gdt al que se debe realizar un JMP FAR para realizar un cambio
de contexto automático.

	Por otra parte antes de realizar el JMP far es importante
verificar si la tarea a la que se va a saltar es una bandera o un
navío. Antes de saltar a una bandera re inicializamos su TSS para
que el EIP vuelva a apuntar a donde es debido, si esto no se hiciera
la ejecución seguiría a partir de la interrupción de fin de bandera
y ese no es el compartamiento esperado. En los navíos, por otra parte
, no hace falta hacer nada en espececial, basta realizar el salto
para que las cosas funcionen.

	Es importante que en el código de la interrupción se contempla
que cuando a la tarea le vuelva a tocar su eip va a estar ubicado adentro
de la interrupción, por lo tanto tiene que siempre haber un JMP al
final de la interrupción, incluso luego de un JMP FAR.
	
	Otra interrupción que controla importantemente el control de
flujo es la interrupción de teclado. De hecho la misma no tiene
ningún control de seguridad. Sencillamente es un gran switch
donde se verifica que tecla se tocó y en caso de caer en alguna
de las teclas contempladas (``e'',``m'',``0-9'') entonces
se realiza el efecto esperado. Las teclas ``e'' y ``m'' se
manejan mediante funciones escritas en C.

	Por último todos los handlers referidos a las excepciones de
sistema (page fault, etc) al final también cambian el flujo de ejecución
pues luego de la lógica referida a la seguridad tienen que retomar
el flujo de ejecución a algún lado, ya sea al lugar donde vino o
, en caso de que esa tarea haya sido desalojada, a algún otro lado
(la tarea idle).

\subsection{Interrupciones de seguridad}

	Las excepciones de sistema como \textit{General Protection}
son el mejor ejemplo. El algoritmo de estas interrupciones es
sencillo, lo que hace es verificar que quién produjo
esa excepción sea efectivamente una tarea. Si fue una tarea la desaloja, 
sino Interrumpe el sistema y avisa que hubo una excepción. Luego
de desalojar la tarea sencillamente salta hacia la tarea idle. Además
todas estas interrupciones tienen exactamente la misma lógica, por lo
que en principio se podrían implementar todas con un macro de nasm.
El único código específico que implementamos para cada una es una funcion de C
que actualiza la pantalla de manera adecuada. Sobre esto se hablará
mas en detalla en la sección de pantalla.

	La syscalls también tienen que hacer algunas verificaciones de
seguridad. Las banderas no pueden llamar a las funciones de los barcos
y los barcos no pueden llamar a la interrupción de final de bandera.

	La interrupción de reloj debe verificar que el contexto en el
que se está ejecutando no sea el de una bandera, pues si es así
debe hacer desalojarla. 



\begin{comment}
	Las interrupciones se manejaron de 2 maneras diferentes:

	Por un lado están las excepciones de sistema. De esas sólo se busca
que si una tarea las produce esta sea desalojada. De esta manera se pudieron
implementar mediante un macro sin mayores inconvenientes.

	Por otro lado están las otras interrupciones: La de reloj, la de teclado
y las syscalls. Estas si se tuvieorn que elaborarse de manera mas detenida.

	La interrupción de teclado basicamente es un gran $switch$ que verifica si
el contenido del código obtenido. Si este coincide con alguna de las teclas esperadas
(``e'',``m'', ``0-9'') entonces realiza la acción debida en pantalla.

	En las syscalls se tiene el cuidado de verificar quién las llama y como. Muchas
posibles llamadas a syscals son ilegales. Por ejemplo si la llamada a una int0x50 la realiza
una bandera esta debe ser desalojada. Lo mismo si esa interrupción es llamada con parámetros ilegales.
Toda esta lógica se implementó medante una función escrita en C y llamada desde assembler.

	La interrupción de reloj terminó siendo la mas compleja. Para implementarla lo que se hizo, entonces,
fue realizar una función en c que se llama desde la interrupción. Esta función verifica si quién
se estaba ejecutando al memento de la interrupción de reloj era una bandera. En ese caso la desaloja.
Luego le pide al scheduller el índice de la GDT que necesario pra realizar un JUMP far al contexto
de la otra tarea. A continuación se fija si la próxima tarea a ejecutarse es una bandera. Si es una bandera
entonces primero reinicia su tss y luego, si, realiza el JMP FAR.
\end{comment}
	
