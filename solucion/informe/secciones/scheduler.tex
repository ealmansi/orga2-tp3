
Dadas las dieciséis tareas del sistema -ocho navíos, ocho banderas-, el scheduler desarrollado realiza un ciclo de ejecución sobre todas las tareas alternando de la siguiente forma: se dedican tres tics de reloj a ejecutar navíos, se procesan todas las funciones bandera pertenecientes a navíos activos, y luego se retoma la ejecución de los navíos. Los navíos se ejecutan secuencialmente por lo que todos reciben la misma cantidad de tics mientras no sean desalojados.

Para implementar esta lógica, se realizaron colas circulares basadas en arreglos para los índices de la tabla GDT de los navíos y las banderas\footnote{De los descriptores de sus tablas TSS, para ser preciso.}. Como no se ingresan nuevos elementos dinámicamente, el mecanismo de elminación de una tarea se puede realizar sencillamente marcando su entrada con un valor inválido, como el 0 ya que nunca es un índice válido en la tabla.

Para mantener el orden de ejecución descripto, se guardaron adicionalmente contadores para la cantidad de navíos activos en la cola, la cantidad de navíos pendientes hasta alternar con la ejecución de banderas, e inversamente la cantidad de banderas pendientes.

De esta forma, se mantiene el invariante de que al menos uno de los dos contadores de tareas pendientes es nulo, y se consumen elementos de la cola opuesta hasta terminar la ronda. Es decir, inicializando la cantidad de navíos pendientes a 3 y la cantidad de banderas pendientes a 0, se dedicarán 3 tics a ejecutar navíos, y luego se asigna la cantidad de tareas activas al contador de banderas pendientes, asegurando que se llamen todas las banderas disponibles.

Como ya se dijo, cuando una tarea necesita ser desalojada, su entrada correspondiente y la de su navío o bandera asociada se anulan, permitiendo evitar su ejecución simplmente salteando los ceros que se encuentren en la cola.

