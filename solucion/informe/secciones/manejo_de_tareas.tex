	El sistema implementado ejecuta un total de hasta 19 tareas en forma itinerante: 
ocho navíos, ocho banderas y la tarea idle. Además, como se dijo previamente, 
los navíos deben comenzar a ejecutarse siempre desde el punto donde fueron interrumpidos, 
mientras que las banderas comienzan siempre desde el inicio de la función. Esto requiere 
mantener registro del estado de los navíos (es decir, guardar todo su contexto para que 
pueda retomar la ejecución sin ser alterada), y un método para intercambiar el contexto 
de diferentes tareas.

	Para resolver este requerimiento, se optó por el cambio automático 
de tareas por hardware que provee la arquitectura Intel x86. Para utilizar este mecanismo, 
es necesario mantener para cada tarea en ejecución una estructura llamada Task State 
Segment (TSS), que consiste en un espacio de memoria donde se guarda (y de donde se 
recupera) el contexto de la tarea de forma automática. Como se dijo en la sección sobre la GDT, 
cada TSS requiere también un descriptor de TSS en la tabla GDT, de forma tal que se pueda 
accionar el mecanismo mediante un \texttt{jmp far}, configurando el selector del salto 
al descriptor TSS de la tarea a la que se desea comenzar a ejecutar.

	Cada vez que se realiza un salto a una nueva tarea, el contexto actual sobreescribe 
la TSS apuntada por el registro Task Register (TR). Esta característica tuvo dos implicaciones 
en la implementación realizada. En primer lugar, previo a la ejecución de la primer tarea 
mediante este mecanismo, el contexto de ejecución no corresponde a ninguna tarea en particular, y 
no tendría sentido que sobreescriba la TSS de ninguna tarea. Por esta razón, se genera una 
TSS adicional para una \"tarea inicial\", cuyo único propósito es ser la TSS que se altera 
al realizar el primer salto. En segundo lugar, como a priori no es necesario guardar el estado de las tareas que ejecutan funciones bandera, se podría prescindir de construir una TSS para cada 
una de ellas. Sin embargo, para proveer un espacio de memoria donde escribir el contexto actual 
cada vez que se interrumpe una tarea tipo bandera, se optó por brindar una TSS a cada una de ellas.

	Para mantener el comportamiento acíclico de las tareas bandera, en vez de recuperar el 
contexto de su TSS asociada (allí se encuentra el estado de la última vez que se ejecutó), se
reinicializa la TSS al igual que la primera vez que fue ejecutada, garantizando que siempre se
procesa desde el comienzo.	
