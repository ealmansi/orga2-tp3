	El manejo de tareas tiene varias aristas. Por un lado está
la gestión ``física'' de las tareas. Es decir como se cambia entre el contexto
de una y el contexto de otra. Este punto se manejo mediante el mecanismo
automático que provee la arquitectura intel.

	Para esto hubo que crear tss para cada tarea. Además por cuestiones
de facilitar la gestión la tareas se decidió crear una tss extra para cada
bandera. De esta forma cada navio tiene 2 tss, una para correr su código
de acción y otra para correr su código de bandera. Además hay 2 tss
extra, una para la tarea idle y otra auxiliar (\emph{tarea inicial}), para poder realizar
el primer salto sin inconvenientes.

	Para poder realizar el primer cambio de contexto (\texttt{JMP FAR}) sobre esta
tss auxiliar, previamente se setea el TR. Este seteo está hardcodeado
porque el descriptor de la tss auxiliar siempre se guarda exactamente
en el mismo índice de la gdt.

	Todos los cambios de tarea se hacen mediante un \texttt{JMP FAR}. En ningún
momento se usa ningún otro mecanismo. El problema que surge con
esto es que las banderas no tienen un código cíclico. Es decir, las banderas
ejecutan una porción de código que termina, al terminar su contexto queda
estático en la posición en la que terminó, por lo tanto si se llama a esa bandera
otra vez en ese mismo contexto sin hacer nada la bandera va a ejecutar 
cosas indebidas. Para subsanar ese inconveniente siempre antes de saltar una
bandera se vuelve a inicializar su tss, de esta forma nos aseguramos de que el contexto
siempre sea el contexto inicial.

	
