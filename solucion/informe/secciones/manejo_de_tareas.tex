	El manejo de tareas tiene varias aristas. Por un lado está
la gestión ``física'' de las tareas. Es decir como se cambia entre el contexto
de una y el contexto de otra. Este punto se manejo mediante el mecanismo
automático que provee la arquitectura intel.

	Para esto hubo que crear tss para cada tarea. Además por cuestiones
de facilitar la gestión la tareas se decidió crear una tss extra para cada
bandera. De esta forma cada navio tiene 2 tss, una para correr su código
de acción y otra para correr su código de bandera.

	Todos los cambios de texto se hacen mediante un JMP FAR. En ningún
momento se usa ningún otro mecanismo. El problema que surge con
esto es que las banderas no tienen un código cíclico. Es decir, las banderas
ejecutan una porción de código que termina, al terminar su contexto queda
estático en la posición en la uqe terminó, por lo tanto se se llama a esa bandera
otra vez en ese mismo contexto sin hacer nada la bandera va a ejecutar 
cosas indebidas. Para
