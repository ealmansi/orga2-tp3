	La arquitectura intel posee varios modos de trabajo. A lo largo
del tp sólo usamos 2 de ellos: El modo real y el modo protegido. Todo
los procesadores con arquitectura intel x86 en el momento en que arrancan lo
hacen en modo real. En modo real el procesador posee la misma interfaz
que un intel 8086. Esto conlleva un tamaño de palabra de 16 bits,
capacidad de dirección de 1mb, nula protección por hardware, etc, etc.

	En modo protegido, por el contrario, el procesador puede hacer pleno
uso de sus recursos. Una vez que se pasa a modo protegido el procesador
puede hacer pleno uso de todo su set de instrucciones, direccionar a 4gb de RAM,
utilizar potentes mecanismos de protección por hardware, etc.

\subsection{Modo real}

	Trabajando en modo real uno está realmente limitado, por lo que
hicimos todo lo posible por salir lo mas rápido posible de el. Sin embarg
algunas cosas se hacen en modo real para ahorrar inconvenientes luego. La
mayoría de estas cosas, sin embargo, fueron implementadas por la cátedra.

	La primera, por supuesto, es correr el bootloader. El bootloader se
ejecuta por completo en modo real. Una vez que el bootloader termina lo siguiente
que se hace es deshabilitar las instrucciones. Muy importante pues todavía
no se definió como manejarlas, por lo que si esto no se hace
en cuanto cae la primera interrupción de reloj se genera una excepción de
la que no se puede salir, pues todavía no se declaró en ninguna parte
que hacer con esa excepción. Luego hay que habilitar A20. Esto 
es buena idea hacerlo antes de saltar a modo protegido, cuando de todas
maneras el límite de direccionamiento es 1 mega. Lo último que se hace antes de pasar
a modo protegido es habilitar los 16 colores de fondo de video. Por defecto
el video está seteado para que el bit mas significativo de las casillas de formato
represente "blink", es decir si el caracter parpadea o no. Lo que hicimos, entonces
, fue setear el video para que ese bit se agruegue a la paleta de colores del fondo
de pantalla. De esta manera se obtiene la posibilidad de crear una pantalla
un poco mas expresiva. Esto decidimos hacerlo aún en modo real principalmente
por el motivo de hacer todo lo relacionado a la comunicación con el hardware
en la etapa mas temprana posible. Es decir "enlistar" el sistema antes de
correrlo.

	Una vez terminadas estas tareas se pasa a modo protegido.

\subsection{Pasaje a modo protegido}

	El pasaje a modo protegido es una parte bastante delicada.
En primer lugar es importante armar una gdt aunque sea minimal y setear
el gdtr de manera adecuada. Sobre eso vamos a entrar en mas detalle
en la sección correspondiente.

	Una vez que esto está asegurado se debe efectivamente pasar
a modo protegido. Esto, en principio, no es mas que setear un bit en el
registro de control CR0. Sin embargo en el momento en que se setea
ese bit cambia la manera de interpretar a los registros de segmento, por lo que
la siguiente instrucción es altamente probable que cause algún problema al
respecto. Lo que se debe hacer entonces es setear los registros de segmento
en valores empezando por el CS, que es el primero que puede producir algún tipo de problema.
Para esto lo que se hace es un jmp far a la siguiente instrucción
hardcodeando el valor adecuado del CS. 

\begin{minted}[tabsize=4, linenos=true, obeytabs=true]{nasm}
	MOV		eax, cr0	; Se trae el contenido de cr0
	OR		eax, 1		; Se cambia el bit correspondiente, el resto quedan iguales
	MOV		cr0, eax	; Se hace realmente el seteo.
	
	JMP		0x90:mp		; La siguiente instruccion debe cambiar el valor del CS, pues sino se
						; va a generar error de proteccion. Para eso se realiza un JMP FAR a 
						; la siguiente instruccion
	BITS 32				; Con esto se le avisa a NASM que a partir de ahora se espera codigo de 32 bits.
mp:						; Esta etiqueta es el destino del JMP FAR, es 
						; decir, la instruccion siguiente al JMP.

; A partir de aca estamos en modo protegido
\end{minted}

\subsection{En modo protegido}

	Ya en modo protegido es importante realizar 2 cosas
antes que nada. Estas son acomodar el resto de los registros
de segmento (el JMP FAR setea el CS, pero faltan todos los demás)
y setear la pila como corresponde. Esto  último consiste sencillamente
en darle a ESP y a EBP los valores adecuados para que la pila quede
donde nosotros queremos (en nuestro caso eso es la posición de memoria 0x27000
de manera tal que la pila viva entre 0x26000 y 0x26FFF.


