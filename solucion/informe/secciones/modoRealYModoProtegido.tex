	En el sistema implementado se utilizaron dos modos de trabajo de los 
múltiples que provee la arquitectura Intel x86: el modo real y el modo protegido. Todos 
los procesadores de dicha arquitectura comienzan a operar en modo real; es decir, 
funcionando similarmente al procesador Intel 8086. Esto conlleva un tamaño de palabra 
de 16 bits, una capacidad de direccionamiento de 1MB, nula protección por hardware, 
entre otras características. Por el contrario, en modo protegido el procesador puede hacer pleno
uso de sus recursos; es decir, su completo set de instrucciones, direccionamiento a 4GB de 
memoria, potentes mecanismos de protección por hardware, y otras ventajas.

\subsection{Modo real}

	Si bien es deseable trabajar en modo protegido por sobre modo real, existe una secuencia de 
pasos que deben ser realizados previamente a realizar la transición. En primer lugar, es necesario 
implementar un bootloader en modo real, el cual se encarga de cargar el código inicial del kernel.

	Una vez que el bootloader finaliza, es necesario deshabilitar las instrucciones. Este paso es 
muy importante pues en esta instancia aún no fue definido el manejo de interrupciones, por lo que
la primera interrupción de reloj generaría una excepción sin capacidad de ser manejada.

	Posteriormente, se requiere la habilitación del gate A20 para permitir direccionamiento a más 
de 1MB. Es razonable hacerlo en modo real, cuando de todas maneras el límite de direccionamiento es 1MB.
Finalmente, previo al paso a modo protegido se habilitó el uso de 16 colores de video. Por defecto
el video está configurado para que el bit mas significativo de las casillas de formato
represente la propiedad "blink", es decir si el caracter parpadea o no. Para lograr una pantalla más 
expresiva, se reconfiguró el sistema para utilizar ese bit como parte del descriptor de color.

	Estos pasos relacionados a la configuración del hardware fueron realizados en modo real principalmente 
para alistar el sistema antes de comenzar la organización del kernel.

\subsection{Pasaje a modo protegido}

	Previo al paso a modo protegido, se debe armar la primer estructura necesaria para 
el funcionamiento del kernel; la Global Descriptor Table (GDT), la cual se discute en más 
detalle en secciones subsiguientes.

	Activar el modo protegido en principio consiste más que en activar el bit menos significativo del 
registro de control CR0. Sin embargo, en el momento en que se setea ese bit, se altera la manera de 
interpretar a los registros de segmento. Para el correcto funcionamiento del sistema en esta etapa, lo que se debe hacer entonces es setear los registros de segmento en valores válidos (según la GDT realizada), 
empezando por el segmento de código (CS), que es el primero que puede producir errores. Para lograr esto,
se realiza un jmp far a la siguiente instrucción, con un valor adecuado para el CS.

	A continuación se incluye un extracto del código mostrando este proceso.

\begin{minted}[tabsize=4, linenos=true, obeytabs=true]{nasm}
	MOV		eax, cr0	; Se trae el contenido de cr0
	OR		eax, 1		; Se cambia el bit correspondiente, el resto quedan iguales
	MOV		cr0, eax	; Se hace realmente el seteo.
	
	JMP		0x90:mp		; La siguiente instruccion debe cambiar el valor del CS, pues sino se
						; va a generar error de proteccion. Para eso se realiza un JMP FAR a 
						; la siguiente instruccion
	BITS 32				; Con esto se le avisa a NASM que a partir de ahora se espera codigo de 32 bits.
mp:						; Esta etiqueta es el destino del JMP FAR, es 
						; decir, la instruccion siguiente al JMP.

; A partir de aca estamos en modo protegido
\end{minted}

\subsection{En modo protegido}

	Una vez en modo protegido, es importante realizar dos tareas en primera instancia: acomodar el 
resto de los registros de segmento (el jmp far configura el CS, pero resta configurar todos los demás)
y setear la pila correctamente; es decir, sencillamente asignar a ESP y a EBP los valores adecuados para que la pila quede donde se haya reservado espacio para ella (en el caso de este sistema, es la posición de memoria 0x27000, alojando la pila 0x26000 y 0x26FFF).


